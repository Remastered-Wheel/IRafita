

Un gran. Mes cultura per a mi:
	Edsger Dijkstra



3, 4, +, 5, 6, +, * => (3 + 4) * (5 + 6)
developed by Dijkstra in 1961

el nom oficial es:
Reverse Polish notation

a notación de postfijo (ej. 2 2 +)

La gran abantatje envers la infijo esk no necessita de parentesis



No se que es per ara:
Shunting-yard algorithm

Es super util
Me transforma d'huma a RPN (notacio polaca inversa)



Vindrie a ser un compilador... Com lo que vull fer a LaTeX o el propi llenguatge que vull crear
Analizador sintáctico
(Redirigido desde «Parser»)


Ara fa temps que no ho llegeixo i no segueixo cap patro ni ordre...
Confio que en un futur si que m'ho currare mes

ordre d'una web
Reglas para Orden de Operaciones

1. Resolver paréntesis, u otros símbolos. ( )  [ ]  { }
2. Resolver exponentes o raíces.
3. Multiplicación y división de izquierda a derecha.
4. Suma y resta de izquierda a derecha. 


Ara nova cosseta pels determinants
Leibniz formula for determinants
pero es de complexitat O(n^3)
aixi k serie millor trobar un altre sistema per a solucionar-ho

Temes a mirar
Análisis_numérico


tambe opino que hauria de comensar de forma seriosa el fet de comensar a fer els diagrames de flux per entendre realment que vull fer
apart d'esser acompanyat per un tant de teoria.
Aixi podria fer coses forza maquetes
